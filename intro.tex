% intro.tex
% Title: Операционные системы. Введение
% Author: Evgeny Simonenko <easimonenko@mail.ru>
% License: CC BY-ND 4.0

\documentclass[9pt,pdf]{beamer}

\usepackage{polyglossia}
\setdefaultlanguage{russian}
\setotherlanguage{english}
\defaultfontfeatures{Ligatures={TeX},Renderer=Basic}
\setmainfont[Ligatures={TeX,Historic}]{Linux Libertine O}
\setsansfont{DejaVu Sans}
\setmonofont{Linux Libertine Mono O}

\usetheme{Verona}
\usecolortheme{dove}

\setcounter{tocdepth}{1}

\usepackage{biblatex}
\bibliography{operating-systems}

\begin{document}
	
\author{Симоненко Е.А.}
\title{Операционные системы}
\subtitle{Введение}
\institute[КубГТУ]{Кубанский государственный технологический университет}
\date{2019}

\nocite{*}
	
\begin{frame}
	\titlepage
\end{frame}
	
\begin{frame}
	\frametitle{Содержание}
	\tableofcontents
\end{frame}

\section{Системное программное обеспечение}

\begin{frame}
	\frametitle{Системное программное обеспечение}
	\begin{quotation}
		Систе́мное програ́ммное обеспе́чение — комплекс программ, которые обеспечивают управление компонентами компьютерной системы, такими как процессор, оперативная память, устройства ввода-вывода, сетевое оборудование, выступая как «межслойный интерфейс», с одной стороны которого аппаратура, а с другой — приложения пользователя. В отличие от прикладного программного обеспечения, системное не решает конкретные практические задачи, а лишь обеспечивает работу других программ, предоставляя им сервисные функции, абстрагирующие детали аппаратной и микропрограммной реализации вычислительной системы, управляет аппаратными ресурсами вычислительной системы.
		
		---
		
		https://ru.wikipedia.org/wiki/Системное\_программное\_обеспечение
	\end{quotation}
\end{frame}

\begin{frame}
	\frametitle{Состав системного программного обеспечения}
	\begin{itemize}
		\item операционные системы
		\item системные утилиты
		\item системы программирования
		\item системы управления базами данных
		\item связующее программное обеспечение
		\item пользовательский интерфейс
	\end{itemize}
\end{frame}

\section{Операционная система}

\begin{frame}
	\frametitle{Операционная система}
	\begin{quotation}
		Операцио́нная систе́ма, сокр. ОС (англ. operating system, OS) — комплекс взаимосвязанных программ, предназначенных для управления ресурсами компьютера и организации взаимодействия с пользователем.
		
		---
		
		https://ru.wikipedia.org/wiki/Операционная\_система
	\end{quotation}
\end{frame}

\begin{frame}
	\frametitle{Структура операционной системы}
	\begin{itemize}
		\item ядро ОС
		\item драйверы устройств
		\item системные сервисы
		\item программный интерфейс (API)
		\item файловая система
	\end{itemize}
\end{frame}

\begin{frame}
	\frametitle{Функции операционной системы}
	\begin{itemize}
		\item управление устройствами компьютера
		\item обеспечение базовых функций хранения и передачи информации
		\item предоставление базового пользовательского интерфейса (не у всех)
		\item управление процессами (не у всех)
		\item управление пользователями (не у всех)
	\end{itemize}
\end{frame}

\section{Классификация операционных систем}

\begin{frame}
	\frametitle{Классификация операционных систем}
	\begin{itemize}
		\item по назначению
		\item по типу компьютера
		\item по типу ядра
		\item по количеству пользователей
		\item по количеству выполняемых задач
		\item по наличию поддержки сети
	\end{itemize}
\end{frame}

\begin{frame}
	\frametitle{Классификация по назначению}
	\begin{itemize}
		\item для пользовательский приложений
		\item для серверных приложений
		\item для управления устройствами
	\end{itemize}
\end{frame}

\begin{frame}
	\frametitle{Классификация по типу компьютера}
	\begin{itemize}
		\item для персональных компьютеров
		\item для серверов
		\item для встраиваемых систем
	\end{itemize}
\end{frame}

\begin{frame}
	\frametitle{Классификация по типу ядра}
	\begin{itemize}
		\item монолитные (Linux, BSD)
		\item микроядерные (GNU/Hurd, OS RT)
	\end{itemize}
\end{frame}

\section{Этапы развития операционных систем}

\begin{frame}
	\frametitle{Этапы развития операционных систем}
	\begin{itemize}
		\item (1945-1955) ОС ещё нет
		\item (1955-1965) ОС пакетной обработки
		\item (1965-1980) многозадачные ОС (OS/360, CTSS, MULTICS, Unix, BSD)
		\item (1975-2010) ОС персональных компьютеров (CP/M, MS-DOS, Windows, Linux)
		\item (1995-2015) ОС мобильных устройств (планшетные ПК, смартфоны)
	\end{itemize}
\end{frame}

\section{Библиография}

\begin{frame}
	\frametitle{Библиография}
	\printbibliography
\end{frame}

\section*{Последний слайд}

\begin{frame}
	\center
	
	\textbf{\textsl{\inserttitle}}
	
	\textsl{\insertsubtitle}
	
	\insertauthor
	
	\url{easimonenko@mail.ru}
\end{frame}

\end{document}
